%!TEX program = xelatex
\documentclass[cn,black,12pt,founder,normal,twocolumn]{elegantnote}
\title{《温病学》沧海拾珠}

\author{shujuecn}

\date{\zhtoday}

\usepackage{array}
\usepackage{ulem}       % 删除线
% \usepackage{geometry}
% \geometry{a4paper,left=1.6cm,right=1.6cm,top=2.5cm,bottom=2.5cm}

\setCJKfamilyfont{SHei}{SimHei}     % 临时黑体
\setCJKfamilyfont{SSong}{SimSun}    % 临时宋体

\setlength{\columnsep}{23pt}
\setlength{\columnseprule}{1pt}

\begin{document}

\maketitle

\section{总论}

温病四大家(P4),温病病因与特点(P20),新感与伏邪温病比较(P23),卫气营血辨证(P28),三焦辨证(P32)。

\subsection{温病诊法}

\subsubsection{辨舌齿}

\begin{enumerate}
    \item 辨舌苔
    \begin{itemize}
        \item 积粉苔:湿热秽浊郁闭膜原。
        \item 白碱苔:宿食与湿热秽浊伏胃。
        \item 白砂苔:邪热化燥伤津。
        \item 白霉苔:胃气衰败,秽浊上泛。
        \item 老黄苔:热结肠腑,阳明腑实。
        \item 灰滑苔:温病后期,阳虚有寒。
        \item 舌淡白无华,苔干黑:热灼血络,气随血脱。
    \end{itemize}
    \item 辨舌质
    \begin{itemize}
        \item 舌光红柔嫩:热退,津伤未复。
        \item 舌淡红干涩:热退,气阴两虚。
        \item 舌纯绛鲜泽:卫分逆传心包,未入气分,伤津不甚。
        \item 舌绛,苔黏腻:热入营血,气分痰湿秽浊不化。
        \item 镜面舌(舌绛无苔):热退阴衰。
        \item 杨梅舌(焦紫起刺):毒炽血分。
        \item 猪肝舌(紫晦而干):肝肾阴竭。
    \end{itemize}
    \item 辨舌体
    \begin{itemize}
        \item 舌体短缩:痰阻舌根,内风扰动。
        \item 舌体萎软:肝肾阴竭,筋脉失养。
    \end{itemize}
\end{enumerate}

\subsubsection{辨斑疹白{\CJKfamily{SHei}㾦}}

\begin{enumerate}
    \item 辨斑疹
    \begin{itemize}
        \item 斑:点大成片,平展皮肤,有形无质,压不退色,消退不脱屑。
        \item 疹:点小成粒,高于皮肤,抚之碍手,压之退色,消退有脱屑。
        \item 色泽:红轻、紫重、黑危。
    \end{itemize}
    \item {辨白\CJKfamily{SSong}㾦}
    \begin{itemize}
        \item 病机:湿热郁阻气分,蕴蒸肌表。
        \item {晶\CJKfamily{SSong}㾦}:晶莹饱满,颗粒清楚。
        \item {枯\CJKfamily{SSong}㾦}:空壳无浆,色如枯骨。
    \end{itemize}
\end{enumerate}

\subsubsection{辨常见症状}

\begin{enumerate}
    \item 发热
    \begin{itemize}
        \item 发热恶寒:温病初起;外感诱发伏邪;暑热内炽阳明。
        \item 寒热往来:湿热郁阻半表半里(少阳、三焦、膜原)。
        \item 身热不扬:湿重于热,湿遏热伏。
        \item 身热夜甚:热入营血,劫灼营阴。
        \item 夜热早凉:邪留阴分,昼退夜争。
    \end{itemize}
    \item 汗出
    \begin{itemize}
        \item 无汗:邪闭肌表;肺气失宣;热灼营阴。
        \item 阵汗:热蒸湿动,湿遏热伏,气机不畅。
        \item 战汗:战后汗出;战后无汗;战后复热;战后热烦。
    \end{itemize}
    \item 小便
    \begin{itemize}
        \item 短少不通(温热):热伤津液,尿源不足。
        \item 短少不通(湿热):气机不畅,水湿内停、偏走膀胱。
        \item 小便涩痛:小肠热盛,下移膀胱。
    \end{itemize}
    \item 大便
    \begin{itemize}
        \item 便稀热臭:肠热下利;热结旁流。
        \item 大便溏垢:湿热积滞,搏结肠腑。
    \end{itemize}
    \item 神智
    \begin{itemize}
        \item 神昏谵语:神志不清,胡言乱语。
        \item 昏聩不语:昏迷不语,呼之不应。
        \item 神识昏蒙:或清或寐,呼之能应。
        \item 神识呆钝:神识淡漠,反应迟钝。
    \end{itemize}
\end{enumerate}

\section{辨证论治}

\subsection{风温}

风温是感受风热病邪所引起的急性外感热病。四季均有,多在冬春季节(12-5月)发病。

\subsubsection{邪袭肺卫}

\begin{itemize}
    \item 银翘散、桑菊饮。
\end{itemize}

\subsubsection{肺热炽盛}

\begin{enumerate}
    \item 邪热袭肺:麻杏石甘汤、千金苇茎汤。
    \item 肺热腑实:宣白承气汤。
    \item 肺热移肠:葛根黄芩黄连汤。
    \item 肺热发疹:银翘散去豆豉,加细生地、丹皮、大青叶,倍玄参方。
\end{enumerate}

\subsubsection{痰热结胸}

\begin{itemize}
    \item 小陷胸加枳实汤。
\end{itemize}

\subsubsection{邪入阳明}

\begin{enumerate}
    \item 热炽阳明:白虎汤。
    \item 热结肠腑:调胃承气汤。
    \item 胃热阴伤:竹叶石膏汤。
\end{enumerate}

\subsubsection{热入心包}

\begin{enumerate}
    \item 热入心包:清宫汤送服安宫牛黄丸、紫雪丹或至宝丹。
    \item 热入心包兼阳明腑实:牛黄承气汤。
\end{enumerate}

\subsubsection{正气外脱}

\begin{itemize}
    \item 生脉散、参附汤。
\end{itemize}

\subsubsection{余邪未尽,肺胃阴伤}

\begin{itemize}
    \item 沙参麦冬汤。
\end{itemize}

\subsection{春温}

春温是感受温热病邪所引起的急性外感热病。春季(3-5月)多发。

\subsubsection{初发证治}

\begin{enumerate}
    \item 气分郁热:黄芩汤加豆豉、玄参方。
    \item 卫气同病:赠损双解散。
    \item 热灼营分:清营汤。
    \item 卫营同病:银翘散去豆豉,加细生地、丹皮、大青叶,倍玄参方。
\end{enumerate}

\subsubsection{邪盛气分}

\begin{enumerate}
    \item 热灼胸膈:凉膈散。
    \item 阳明热盛:白虎汤。
    \item 热结肠腑:增液承气汤、新加黄龙汤或导赤承气汤。
\end{enumerate}

\subsubsection{热燔营血}

\begin{enumerate}
    \item 气营(血)两燔:玉女煎去牛膝、熟地黄加生地黄、玄参方,或化斑汤、清瘟败毒饮。
    \item 热盛动血:犀角地黄汤。
    \item 热与血结:桃仁承气汤。
\end{enumerate}

\subsubsection{热陷心包}

\begin{enumerate}
    \item 邪热闭窍:清宫汤送服安宫牛黄丸、紫雪丹或至宝丹。
    \item 内闭外脱:生脉散、参附汤合安宫牛黄丸、紫雪丹或至宝丹。
\end{enumerate}

\subsubsection{阳气暴脱}

\begin{itemize}
    \item 参附汤、回阳救逆汤。
\end{itemize}

\subsubsection{热盛动风}

\begin{itemize}
    \item 羚角钩藤汤。
\end{itemize}

\subsubsection{热灼营阴}

\begin{enumerate}
    \item 真阴亏损:加减复脉汤。
    \item 阴虚动风:三甲复脉汤、大定风珠。
    \item 阴虚火炽:黄连阿胶汤。
\end{enumerate}

\subsubsection{邪留阴分}

\begin{itemize}
    \item 青蒿鳖甲汤。
\end{itemize}

\subsection{暑温}

暑温是感受暑热病邪引起的一种急性外感热病。夏暑季节(6--8月)多发。

\subsubsection{暑温本病}

\begin{enumerate}
    \item 气分证治
    \begin{itemize}
        \item 暑入阳明:白虎(加人参)汤。
        \item 暑伤津气:王氏清暑益气汤。
        \item 津气欲脱:生脉散。
        \item 热结肠腑:解毒承气汤。
    \end{itemize}
    \item 营血分证治
    \begin{itemize}
        \item 暑入心营:清营汤送服安宫牛黄丸、紫雪丹或至宝丹。
        \item 气营两燔:玉女煎去牛膝、熟地黄加生地黄、玄参方。
        \item 暑热动风:羚角钩藤汤。
        % \item \sout{暑入血分:神犀丹、安宫牛黄丸}。
        \item 暑伤肺络:犀角地黄汤合黄连解毒汤。
    \end{itemize}
    \item 后期证治
    \begin{itemize}
        \item 暑伤心肾:连梅汤。
        \item 暑热未尽,痰瘀滞络:三甲散。
    \end{itemize}
\end{enumerate}

\subsubsection{暑温兼证}

\begin{enumerate}
    \item 暑湿在卫:卫分宣湿饮、新加香薷饮。
    \item 暑湿困阻中焦:白虎加苍术汤。
    \item 暑湿弥漫三焦:三石汤。
    \item 暑湿伤气:东垣清暑益气汤。
    % \item \sout{暑湿未净,蒙扰清阳:清络饮}。
\end{enumerate}

\subsection{湿温}

湿温是由湿热病邪所引起的一种急性外感热病。长夏之季(7-8月)多发。

\subsubsection{湿重于热}

\begin{enumerate}
    \item 湿遏卫气:藿朴夏苓汤、三仁汤。
    \item 邪阻膜原:达原饮、雷氏宣透膜原法。
    \item 邪困中焦:雷氏芳香化浊法。
    \item 湿阻肠道,传导失司:宣清导浊汤。
    \item 湿浊上蒙,泌别失职:苏合香丸、茯苓皮汤。
\end{enumerate}

\subsubsection{湿热并重}

\begin{enumerate}
    \item 湿热困阻中焦:王氏连朴饮。
    \item 湿热蕴毒:甘露消毒丹。
    \item 湿热酿痰,蒙蔽心包:菖蒲郁金汤。
\end{enumerate}

\subsubsection{热重于湿}

\begin{itemize}
    \item 白虎加苍术汤。
\end{itemize}

\subsubsection{化燥入血}

\begin{itemize}
    \item \sout{犀角地黄汤加减}。
\end{itemize}

\subsubsection{湿从寒化}

\begin{itemize}
    \item 四加减正气散、五加减正气散。
\end{itemize}

\subsubsection{后期证治}

\begin{enumerate}
    \item 湿盛阳微:薛氏扶阳逐湿汤、真武汤。
    \item 余邪未尽:薛氏五叶芦根汤。
\end{enumerate}

\subsection{伏暑}

伏暑是由暑湿或暑热病邪伏藏,为秋冬时令之邪所诱发的一种急性外感热病。深秋或冬季(11-2月)发病。

\subsubsection{初发证治}

\begin{enumerate}
    \item 卫气同病:银翘去牛蒡子、玄参加杏仁、滑石方,或黄连香薷饮。
    \item 卫营同病:银翘散加生地黄、丹皮、赤芍、麦冬方。
\end{enumerate}

\subsubsection{邪在气分}

\begin{enumerate}
    \item 暑湿郁阻少阳:蒿芩清胆汤。
    \item 暑湿夹滞,搏结肠腑:枳实导滞汤。
    \item 热炽阴伤:冬地三黄汤。
\end{enumerate}

\subsubsection{热在营血}

\begin{enumerate}
    \item 热在心营,下移小肠:导赤清心汤。
    \item 热闭心包,血络瘀滞:犀地清络饮。
    \item 热瘀气脱:犀角地黄汤。
\end{enumerate}

\subsubsection{肾气亏损,固摄失司}

\begin{itemize}
    \item 右归丸合缩泉丸。
\end{itemize}

\subsection{秋燥}

\subsubsection{邪犯肺卫}

\begin{itemize}
    \item 桑杏汤。
\end{itemize}

\subsubsection{邪在气分}

\begin{enumerate}
    \item 燥干清窍:翘荷汤。
    \item 燥热伤肺:清燥救肺汤。
    \item 肺燥肠热,络伤咳血:阿胶黄芩汤。
    \item 腑实阴伤:调胃承气汤加鲜首乌、鲜生地黄、鲜石斛。
    \item 肺燥肠闭:五仁橘皮汤。
    \item 肺胃阴伤:沙参麦门冬汤、五汁饮。
\end{enumerate}

\subsubsection{气营(血)两燔}

\begin{itemize}
    \item 玉女煎去牛膝、熟地,加生地、玄参。
\end{itemize}

\subsubsection{燥伤真阴}

\begin{itemize}
    \item 三甲复脉汤、小复脉汤。
\end{itemize}

\subsection{大头瘟}

大头瘟是感受风热时毒引起的,以头面红肿热痛为特征的急性外感热病。多发于冬春季节。

\begin{enumerate}
    \item 邪犯肺卫:葱豉桔梗汤、如意金黄散。
    \item 毒壅肺胃:普济消毒饮、三黄二香散。
    \item 毒炽肺胃,热结肠腑:通圣消毒散、三黄二香散。
    \item 胃阴耗伤:七鲜育阴汤。
\end{enumerate}

\subsection{烂喉痧}

烂喉痧是外感温热时毒而引起的急性外感热病。多发于冬春二季。

\begin{enumerate}
    \item 毒侵肺卫:清咽栀豉汤、玉豉散。
    \item 毒壅气分:余氏清心凉膈散、锡类散。
    \item 毒燔气营血:凉营清气汤、珠黄散。
    \item 余毒伤阴:清咽养营汤。
\end{enumerate}

\subsection{瘟疫}

瘟疫是感受疫疠病邪所引起,以起病急骤,传遍迅速,病情凶险,传染流行性强为主要特征的一类急性外感热病。

\begin{enumerate}
    \item 卫气同病:赠损双解散。
    \item 温热疫邪充斥三焦:升降散。
    \item 湿热疫毒阻遏膜原:达原饮。
    \item 阳明热炽,迫及营血:化斑汤。
    \item 邪毒炽盛,气营(血)两燔:清瘟败毒饮。
    \item 血分实热,血热妄行:犀角地黄汤。
    \item 毒陷心包,肝风内动:清宫汤合羚角钩藤汤。
    \item 正气暴脱:生脉散、四逆汤。
    \item 正衰邪恋:三甲散。
\end{enumerate}

\subsection{疟疾}

疟疾是感受疟邪引起的以寒战、壮热、头痛、汗出及休作有时为主要特征的急性外感热病。多发于夏秋季节。

\begin{enumerate}
    \item 正疟:小柴胡汤。
    \item 温疟:白虎加桂枝汤。
    \item 暑疟:蒿芩清胆汤。
    \item 湿疟:厚朴草果汤。
    \item 寒疟:柴胡桂枝干姜汤。
    \item 瘴疟
    \begin{itemize}
        \item 热瘴:清瘴汤。
        \item 冷瘴:不换金正气散。
    \end{itemize}
    \item 劳疟:何人饮。
    \item 疟母:鳖甲煎丸。
\end{enumerate}

\subsection{霍乱}

霍乱是感受时行秽浊疫疠之邪,随饮食侵入人体胃肠而引起的,以上吐下泻为主要特征的一种急性疾病。

\begin{enumerate}
    \item 湿热证:蚕矢汤、燃照汤。
    \item 寒湿证:藿香正气散、附子理中丸。
    \item 干霍乱:玉枢丹、行军散加减。
    \item 亡阴:生脉散、大定风珠。
    \item 亡阳:通脉四逆加猪胆汁汤、参附汤。
\end{enumerate}

\section{方证症鉴别}

\subsection{银翘散—桑菊饮}

\subsubsection{邪袭肺卫—银翘散证(P71)}

\begin{enumerate}
    \item 病因:感受风热病邪较重。
    \item 病机:邪犯卫表,肺失开阖。
    \item 症状:发热较重,咳嗽较轻。
    \item 治法:辛凉解表,宣肺泄热。
    \item 方剂:\uline{银翘散}。
\end{enumerate}

\subsubsection{邪袭肺卫—桑菊饮证(P71)}

\begin{enumerate}
    \item 病因:感受风热病邪较轻。
    \item 病机:邪犯卫表,肺失宣肃。
    \item 症状:发热较轻,咳嗽较重。
    \item 治法:辛凉解表,宣肺止咳。
    \item 方剂:\uline{桑菊饮}。
\end{enumerate}

\subsection{承气汤证}

% 主症为痞、满、燥、实、坚,可兼夹湿浊、瘀血、热毒或气津耗伤等症。

\subsubsection{风温—肺热腑实(P73)}

\begin{enumerate}
    \item 病机:肺经痰热郁阻,肠腑热结不通。
    \item 主症:潮热,便秘,喘促,痰多。
    \item 治法:宣肺化痰,泄热攻下。
    \item 方剂:\uline{宣白承气汤}。
\end{enumerate}

\subsubsection{风温—热结肠腑(P75)}

\begin{enumerate}
    \item 病机:肺热不解,传入胃肠。
    \item 主症:潮热,腹痛,便秘或热结旁流。
    \item 治法:软坚攻下泄热。
    \item 方剂:\uline{调胃承气汤}。
\end{enumerate}

\subsubsection{春温—热结肠腑(P86)}

\begin{enumerate}
    \item \uline{增液承气汤}
    \begin{itemize}
        \item 病机:阳明腑实,阴液亏虚。
        \item 主症:口干唇燥,舌苔干燥。
        \item 治法:攻下腑实,滋阴增液。
    \end{itemize}
    \item \uline{新加黄龙汤}
    \begin{itemize}
        \item 病机:阳明腑实,气液两虚。
        \item 主症:口干咽燥,怠倦少气。
        \item 治法:攻下腑实,滋阴益气。
    \end{itemize}
    \item \uline{导赤承气汤}
    \begin{itemize}
        \item 病机:阳明腑实,小肠热盛。
        \item 主症:小便红赤,涩痛不畅。
        \item 治法:攻下腑实,清泄小肠。
    \end{itemize}
\end{enumerate}

\subsubsection{春温—热与血结(P89)}

\begin{enumerate}
    \item 病机:热与血结,瘀血蓄积下焦。
    \item 主症:身热腹满痛,躁狂,舌绛脉涩。
    \item 治法:泄热通结,活血逐瘀。
    \item 方剂:\uline{桃仁承气汤}。
\end{enumerate}

\subsubsection{暑温—热结肠腑(P99)}

\begin{enumerate}
    \item 病机:暑热伤津,热结阳明。
    \item 主症:痞满燥实坚,谵狂,舌卷囊缩。
    \item 治法:通腑泄热,清热解毒。
    \item 方剂:\uline{解毒承气汤}。
\end{enumerate}

\subsubsection{伏暑—湿热夹滞,搏结肠腑(P127)}

\begin{enumerate}
    \item 病机:暑湿郁蒸气分,搏结积滞。
    \item 主症:稽留热,便溏不爽,色黄如酱。
    \item 治法:导滞通下,清热化湿。
    \item 方剂:\uline{枳实导滞汤}。
\end{enumerate}

\subsection{麻杏石甘汤—清燥救肺汤}

\subsubsection{风温—麻杏石甘汤证(P72)}

\begin{enumerate}
    \item 病机:风热病邪入里,邪热壅肺。
    \item 主症:咳喘,痰黄稠、铁锈色、带血。
    \item 特点:津伤不明显,属邪实。
    \item 治法:清热宣肺。
\end{enumerate}

\subsubsection{秋燥—清燥救肺汤证(P135)}

\begin{enumerate}
    \item 病机:肺经燥热化火,耗伤阴液。
    \item 主症:咳喘,无痰,鼻喉齿干燥。
    \item 特点:津伤明显,实中有虚。
    \item 治法:清肺润燥养阴。
\end{enumerate}

\subsection{黄连阿胶汤—连梅汤}

\subsubsection{春温—黄连阿胶汤证(P92)}

\begin{enumerate}
    \item 病机:邪热久稽,肾阴伤,心火亢盛。
    \item 主症:心烦不寐,舌红,脉细数。
    \item 特点:火旺,心烦不寐为主。
    \item 治法:清热降火,育阴安神。
\end{enumerate}

\subsubsection{暑温—连梅汤证(P102)}

\begin{enumerate}
    \item 病机:暑热久稽,肾阴伤,水火不济。
    \item 主症:心烦燥热,消渴不已。
    \item 特点:阴伤,消渴为主。
    \item 治法:清心泻火,滋肾养阴。
\end{enumerate}

\subsection{白虎加人参—王氏清暑益气}

\subsubsection{暑温—白虎加人参汤(P97)}

\begin{enumerate}
    \item 病机:暑热阳明,多汗耗伤津气。
    \item 主症:热渴汗出脉洪大,背微恶寒。
    \item 特点:暑热较重,津气耗伤较轻。
    \item 治法:清泄暑热,益气生津。
\end{enumerate}

\subsubsection{暑温—王氏清暑益气汤(P98)}

\begin{enumerate}
    \item 病机:暑热亢盛,津气两伤。
    \item 主症:烦热,气短促,身疲肢倦。
    \item 特点:暑热较轻,津气耗伤较重。
    \item 治法:益气生津,清热涤暑。
\end{enumerate}

\subsection{辛凉—甘寒—酸泻酸敛}

\subsubsection{首用辛凉(P97)}

\begin{itemize}
    \item 方剂:\uline{白虎(加人参)汤}。
    \item 治法:清泄暑热。
\end{itemize}

\subsubsection{继用甘寒(P98)}

\begin{itemize}
    \item 方剂:\uline{王氏清暑益气汤}。
    \item 治法:益气生津,清热涤暑。
\end{itemize}

\subsubsection{再用酸泻酸敛(P97)}

\begin{enumerate}
    \item \uline{生脉散}:益气敛津,扶正固脱。
    \item \uline{连梅汤}:清心泻火,滋肾养阴。
\end{enumerate}

\subsection{暑厥—暑风—暑瘵}

\subsubsection{暑厥(P100)}

\begin{enumerate}
    \item 证候:暑入心营。
    \item 病机:暑入气分不解,内传心营;或暑邪猝中心营,内闭心包。
    \item 主症:心烦谵语不寐,舌蹇绛,肢厥。
    \item 治法:清营泄热,清心开窍。
    \item 方剂:\uline{清营汤送服三宝}。
\end{enumerate}

\subsubsection{暑风(P101)}

\begin{enumerate}
    \item 证候:暑热动风。
    \item 病机:暑热亢盛,引动肝风。
    \item 主症:四肢抽搐,角弓反张。
    \item 治法:清泄暑热,息风定痉。
    \item 方剂:\uline{羚角钩藤汤}。
\end{enumerate}


\subsubsection{暑瘵(P102)}

\begin{enumerate}
    \item 证候:暑伤肺络。
    \item 病机:暑热犯肺,损伤血络。
    \item 主症:烦渴,咳喘,咳血或血丝痰。
    \item 治法:凉血解毒,清暑安络。
    \item 方剂:\uline{犀角地黄汤}合\uline{黄连解毒汤}。
\end{enumerate}

\subsection{凉开三宝}

\begin{enumerate}
    \item \uline{安宫牛黄丸}(P77)
    \begin{itemize}
        \item 药性:最凉,长能清热解毒。
        \item 主症:高热昏迷。
    \end{itemize}
    \item \uline{紫雪丹}(P77)
    \begin{itemize}
        \item 药性:偏凉,长能息风止痉。
        \item 主症:高热惊厥。
    \end{itemize}
    \item \uline{至宝丹}(P77)
    \begin{itemize}
        \item 药性:芳香辟秽,开窍醒神。
        \item 主症:窍闭谵语。
    \end{itemize}
\end{enumerate}

\section{方剂歌诀}

\begin{enumerate}
    \item \uline{千金苇茎汤}:\\
    千金苇茎桃薏冬,清肺化痰逐肺痈。
    \item \uline{竹叶石膏汤}:\\
    竹叶石膏汤人参,麦冬半夏甘草临。\\
    再加粳米同煎服,清热益气养阴津。
    \item \uline{沙参麦冬汤}: \\
    沙参麦冬扁豆桑,玉竹花粉甘合方。\\
    秋燥耗伤肺胃液,苔光干咳此堪尝。
    \item \uline{凉膈散}:\\
    凉膈硝黄山栀翘,黄芩甘草薄荷饶。
    \item \uline{羚角钩藤汤}:\\
    俞氏羚角钩藤汤,桑菊茯神鲜地黄。\\
    芍草川贝鲜竹茹,凉肝增液定风方。
    \item \uline{蒿芩清胆汤}:\\
    蒿芩清胆枳竹茹,陈夏茯苓加碧玉。
    \item \uline{葱豉桔梗汤}:\\
    葱豉桔梗山栀翘,薄荷竹叶甘草饶。\\
    风温身热咳咽痛,疏风清热症可消。
    \item \uline{如意金黄散}:\\
    如意黄金散大黄,姜黄柏芷陈皮苍。\\
    南星厚朴花粉草,敷之肿胀可安康。
    \item \uline{普济消毒饮}:\\
    普济消毒蒡芩连,甘桔蓝根勃翘玄。\\
    升柴陈薄僵蚕入,大头瘟毒服之痊。
    \item \uline{三黄二香散}:\\
    大黄连柏共三黄,乳香没药为二香。\\
    研末茶油继调敷,头面毒肿药效强。
\end{enumerate}




















% \begin{table}[htbp]
%   \centering
%   \small
%   \caption{燃油效率与汽车价格}
%     \begin{tabular}{lcc}
%     \toprule
%                   &       (1)         &        (2)      \\
%     \midrule
%     燃油效率      &   -238.90***      &      -49.51     \\
%     \bottomrule
%     \end{tabular}%
%   \label{tab:reg}%
% \end{table}%

% \begin{figure}[htbp]
%   \centering
%   \includegraphics[width=\textwidth]{star.png}
%   \caption{一键三连求赞}
% \end{figure}

\end{document}
